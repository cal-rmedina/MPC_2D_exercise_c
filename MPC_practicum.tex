\documentclass[sizes,12pt,nftimes]{article}
%\documentstyle[sizes,12pt]{article}
\setlength{\textheight}{21.0cm} \setlength{\textwidth}{17.0 cm}
%\setlength{\paperwidth}{21.0cm}
\setlength{\oddsidemargin}{-0.5cm}
\setlength{\evensidemargin}{0.5cm} \setlength{\topmargin}{-0.5cm}

\usepackage{graphicx}
\usepackage{amssymb}
\usepackage{bm}
\usepackage{epsfig}
\usepackage{graphicx}
\usepackage{times}
\usepackage{float}
\usepackage{natbib}
%\renewcommand{\bibsection}{}
\usepackage[usenames,dvipsnames]{color}
\usepackage{xcolor}
\usepackage{todonotes}

\voffset -1cm
\oddsidemargin 0.2cm
\textheight 23. cm

\begin{document}

\begin{center}
{\sc Initial Training on Numerical Methods for Active Matter,} 18 Jan - 4 Feb 2021 

\vspace{1cm}
{\large {\sc Hands-on exercise for the lecture} }\\[1ex]
 
{\large {\bf Multiparticle Collision Dynamics:} } \\ [1ex]
{\large {\bf A method to unravel 
  the hydrodynamic properties of active matter}} \\ [1ex]

%\vspace{0.5cm}
Alberto Medina, Joscha Mecke, Marisol Ripoll \\ [1ex]
Theoretical Physics of Living Matter, Institute of Biological Information Processing, 
Forschungszentrum J\"ulich, 52425 J\"ulich, Germany \\[1ex]
January 28, 2021 \\
\end{center}

\section*{Introduction} 
This is an introductory exercise to the multiparticle collision
dynamics (MPC) simulation technique. This text and the exercise are
complementary to the lecture, to the corresponding notes, and to all
the references there included.  The exercise tries to adapt to the
different background of the PhD students attending this course, and to
any other potential interested reader. Some parts of the exercise are
therefore simpler, and some others more involved, but still
accesible. We leave the deeper understating of the code details to the
most interested and expert participants.  The main idea is that the
principles of the method are revised and become more clear. {\em We
  hope you enjoy it.}

\vspace{0.5cm}
Note that: \hspace{0.5cm} {\bf To do this exercise, you need a C compiler
 and python 2 or higher} 

C compilers: gcc (Unix), Visual Studio (Win), trial version of intel icc (Win/Unix), ...

\subsection*{Physical frame}
The code coming with this exercise contains a simple but complete MPC
program in two-dimensions. The idea is to study the fluid behaviour in
a microchannel geometry, for which periodic boundary conditions in
$x$-direction and walls in $y$-direction at $y=0$ and $y=L_y$ are
considered. To generate fluid in motion, a constant force $g$,
``gravity'', is applied to all particles in the direction parallel to
the walls.

The MPC method is a mesoscopic approach designed to simulate fluid
hydrodynamics. The transport properties can be analytically calculated
by means of kinetic theory (see the lecture notes). In particular,
viscosity is the sum of two contributions, $\eta = \eta_\mathrm{kin} +
\eta_\mathrm{col}$. The kinetic contribution $\eta_\mathrm{kin}$
refers to the momentum transfer due to the translation of the
particles, while the collisional contribution $\eta_\mathrm{coll}$
refers to the momentum transferred over multiple particle collisions. 
In two dimensions these expressions are,
\begin{eqnarray}
    \eta_\mathrm{kin} &=& \frac{\rho k_\mathrm{B}T h}{a^2} \left\{ 
       \frac{\rho/\left( 1 - \cos 2\alpha \right)}{\rho - 1 + e^{-\rho}} - \frac{1}{2} \right\}, 
        \nonumber \\ \label{eq:eta}
    \eta_\mathrm{col} &=& \frac{m \left( 1 - \cos \alpha \right) \left( \rho - 1 + e^{-\rho} \right)}{12 h},
\end{eqnarray}
where $h$ is the MPC collision time, $\rho$ is the fluid density,
$\alpha$ is the rotation angle, $a$ is the collision box length, $m$
is the particle mass and $k_\mathrm{B}T$ is the thermal energy
(typically, the collision box length, particle mass, and system
average temperature are chosen to be the systems units, \emph{i.e.},
$a = 1 = m = {\mathrm k}_B T$). Note that the derivation of the
transport properties required some approximations, as discussed in the
lecture notes.
  
The two dimensional fluid flow in the microchannel satisfies the
Navier-Stokes equation, which in the case of small Reynolds number
(small velocities and laminar flow) reduces to the {\em Stokes
equation}, and in this case further simplifies to
  \begin{equation}
    \label{eq:NSE}
    \partial_y^2 v_x = \frac{1}{\eta} g,
  \end{equation}
once the system has reached the steady-state. Hence, the resulting
  flow field $\bm{v}(\bm{x})$ is constant along the $x$-direction and
  only depends on the $y$-direction. Integrating
  equation~eq.~(\ref{eq:NSE}) and fixing the integration constants
  with the no-slip boundary conditions $v_x(0)=0$ and $v_x(L_y)=0$
  yields a parabolic profile. Exercises 1 and 2 you will check  among other things, these 
profiles and their relation with the viscosity values. 
  
\vspace*{0.2cm} An important requirement for the simulation method is
how to couple the interaction of a solute (typically large object) to
the fluid, \emph{e.g.}, a colloid or a polymer.  In the lecture we
have seen the most standard methods, which is to consider the object
as a heavy MPC particle, or to interact with Molecular dynamics or the
surface boundary. In the last part of exercise 1 we propose an interesting
alternative method, which on a microscopic level, might be regarded as
unphysical, although on a coarse-grained level, yields the desired
behaviour.
  
\subsection*{Technical code comments }

Exercise is available on \mbox{\bf https://github.com/cal-rmedina/MPC\_2D\_exercise\_c} 
 
\noindent If you've modified the code and it doesn't work you can always download the pristine version.
 
\vspace*{0.2cm}
The simulation program can be compiled and run with \mbox{\texttt{Makefile}}, 
if you are not familiar with \mbox{\texttt{make}} commands you can also compile
with \mbox{\texttt{gcc mpc\_2d.c -lm -o mpc\_2d.exe}}. 
Once you have the executable,it can be run as \mbox{\texttt{./mpc\_2d.exe}}. 
The parameters are read in from the file \mbox{\texttt{system\_parameters.h}}.
You do not have to recompile after changing the parameters, but only after changing
the source code. For modifying the input parameters and the source code,
you can use whatever text editor you wish. 

  
The simulation results are visualized with python using the plotting
library \mbox{\texttt{matplotlib}}. You may find prepared plotting scripts in the
folder \mbox{\texttt{output/}}. Once your simulation finishes, you can
plot the results by typing \mbox{\texttt{./plot\_flowprofile.py}} or
\mbox{\texttt{./plot\_flowfield.py}}. The graphs will then be saved to
\mbox{\texttt{flowprofile.pdf}}, \mbox{\texttt{flowfield.pdf}},
respectively. If this file already exists then it will be overwritten,
so you need to rename them if you would like to save them
permanently. If you change the system properties (channel size, fluid
density, etc.) in the input file, you will have to edit these files as
well in the python scripts so that the visualisation follows suit.
  
\newpage
\section*{Exercises}
 
\paragraph*{Exercise 1} The idea in this exercise is to become more familiar
with the method and the related variables. In this way, you only need
to modify the input parameters to understand the fluid behaviour.

\begin{enumerate}
\item Compile and run the simulation. Plot the flow field and the flow
  profile. Is the latter parabolic?
\item Experiment on changing the equilibration time and the total
  simulation time. What happens?
\item Measure the velocity profile for different fluid densities, time
  steps, and applied external forces. The theoretical profile is here
  measured with eq.~(\ref{eq:eta}). When do and when do not the
  measured and theoretical profiles agree? Why?
\item Switch off the grid shift by editing the grid-shift parameter in
  \mbox{\texttt{system\_parameters.h}} and run the simulation
  again. What happens?
\end{enumerate}

In the second part of this exercise, we model an immobile obstacle, \emph{e.g}, a fixed colloid, by considering a number of heavy fixed fluid particles with velocities randomly drawn from a Maxwell-Boltzmann distribution before each collision to ensure a vanishing mean velocity of the fixed colloid. The heavy fluid particles take part in the collision step, resulting in a very high friction in the obstacle region. During the streaming step, the fluid particles may
still enter the obstacle.

\begin{enumerate} \addtocounter{enumi}{4}
\item Add an obstacle to the flow by editing the respective value in
  \mbox{\texttt{system\_parameters.h}} to, \emph{e.g.}, $5$. Plot the
  flow field and the flow profile. What happens?  Try different radii.
\item With an obstacle present, turn off the grid-shift and rerun the
  simulation. What happens? Can you find the physical explanation
  for these different behaviors ?
\end{enumerate}


\paragraph*{Exercise 2} The idea now is to understand a bit more in depth the
method and the code, such that you also have to modify the source code
of the simulation, compile and run. 
		
An unfortunate researcher attempted to optimize the code a little bit,
but got something wrong. It is your task to fix it in this
exercise. Each time that you change the simulation code you have to
recompile. (Note the hints at the end of the exercise)
		
\begin{enumerate}
\item Look at the source code, study its structure and identify where
  the MPC streaming and collision steps take place. The called
  functions \mbox{\texttt{stream()}} and
  \mbox{\texttt{collide()}} can be found in the file
  \mbox{\texttt{modules/mpc\_routines/mpc\_routines.h}}.
\item The collision step is broken: The velocities in the given cell
  always rotate counter-clockwise. What we want is that within a given
  collision cell, the velocities rotate in the same direction. Whether
  this is clockwise or counter-clockwise for a given cell should be
  randomly chosen, both with equal probability. Fix the collision
  step, recompile and run the simulation and look at the flow profile.
\item The fluid is now at thermal equilibrium, this is just
  resting. Boring ? Implement a gravity-type of force: A force,
  proportional to the global variable \mbox{\texttt{grav}} should
  accelerate every fluid particle in the positive $x$-direction. How
  does this change the flow profile?
\item The friction between the fluid and the wall is not quite OK. If
  you look at the source code, you will see that the fluid particles
  bounce off the walls like billiard balls bounce off the side of the
  pool table, so on average they move fast close to the wall. This is
  the slip boundary condition. What we want is the no-slip boundary
  condition, in which the average velocity of the fluid close to the
  wall should be zero. For this, the momentum component of the
  bouncing fluid particles parallel to the wall should also be
  reversed, \emph{i.e.}, the fluid particles fly back towards the
  direction they were coming from. Fix the boundary condition and
  study the flow profile.
\item The correct fluid-particle bouncing is not enough to produce a
  no-slip boundary condition. Also so-called virtual wall particles
  are needed. They have already been implemented but a final
  \mbox{\texttt{\#define VIRTUAL\_PARTICLES}} is missing in
  \mbox{\texttt{mpc\_2d.c}} before the main function. Find the part of
  code implementing the virtual particles, and study it briefly. What
  does it seem to do and how does this affect the flow profile?
\item Now everything should be fixed. Run a couple of simulations with
  the parameters you used in exercise 1 to check that everything works
  fine. To this end, compare your results with those of exercise 1.
\end{enumerate}

\begin{description}
\item [Hint 1] In 2 (b), compare the two-dimensional rotation matrices
  which cause a rotation counter-clockwise and clockwise about an
  angle $\alpha$ or $-\alpha$.
		
  \begin{equation}
    \bm{R}_\mathrm{ccw}(\pm\alpha) = \left(
\begin{tabular}{rr} $\cos \alpha$ & $\mp \sin \alpha$ \\
     $\pm \sin \alpha$ & $\cos \alpha$
    \end{tabular} \right)
  \end{equation}
  
 The macro \texttt{RND1} gives you a random floating
 point number in the range $[0.0, 1.0)$. The main
  function \texttt{main()} can be found in
                \texttt{mpc\_2d.c}.
\item [Hint 2] In 2 (c), you have to modify the
                streaming step. A standard integration scheme is the
                so-called Velocity-Verlet integration. The scheme is
                as follows
\begin{enumerate}
  \item Calculate $\bm{v}(t + \frac{\delta t}{2}) = \bm{v}(t) + \frac{1}{2} \bm{a}(t) \delta t$
  \item Calculate $\bm{x}(t + \delta t) = \bm{x}(t) + \bm{v}(t + \frac{\delta t}{2}) \delta t$
  \item Calculate $\bm{a}(t + \delta t)$ form the interactions
  \item Calculate $\bm{v}(t + \delta t) = \bm{v}(t + \frac{\delta t}{2}) + \frac{1}{2} \bm{a}(t + \delta t) \delta t$\,.
\end{enumerate}
  This can be significantly simplified considering that the acceleration $\bm{a}(t)$ is constant 
\end{description}

\subsection*{Acknowledgments}
The very first version of this exercise was prepared already in 2006
by Marisol Ripoll, and since them several generation of students of
the group in J{\"u}lich have used, iteratively strongly modified, and
very much improved it until its present, and most likely not yet final
form. Thanks to all those that already helped and to you for your
interest. Any suggestion you might have for further improvement is
also most welcome.

\end{document}